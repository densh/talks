
% The following \documentclass options may be useful:
%
% 10pt          To set in 10-point type instead of 9-point.
% 11pt          To set in 11-point type instead of 9-point.
% authoryear    To obtain author/year citation style instead of numeric.

\usepackage{amsmath}
\usepackage{amssymb}

%\usepackage{fleqn}
\usepackage{listings}
\usepackage{math}
\usepackage{amsmath}
\usepackage{latexsym}
\usepackage{bcprules}
%\usepackage[scaled=0.848971]{luximono} % This is for 11 pt Default font
\usepackage[T1]{fontenc}

% Prooftree formatting
\usepackage{prooftree}

\usepackage{multicol}
\usepackage{framed}

\usepackage{stmaryrd}

%\usepackage{float}
%\floatstyle{boxed}
%\restylefloat{figure}

% support for generating PDF files
%\newif\ifpdf
%    \ifx\pdfoutput\undefined
%    \pdffalse
%\else
%    \pdftrue
%    \pdfoutput=1
%\fi

%versions
% Use dependent function types
\newif\ifdep\depfalse

\lstset{
  literate=
  {=>}{$\Rightarrow\;$}{2}
  {<:}{$<:\;$}{1}
}

\lstdefinelanguage{scala}{%
       morekeywords={%
                try, catch, throw, private, public, protected, import, package, implicit, final, package, trait, type, class, val, def, var, if, for, this, else, extends, with, while, new, abstract, object, case, match, sealed,override},
         sensitive=t, %
   morecomment=[s]{/*}{*/},morecomment=[l]{\//},%
   mathescape,
%   escapeinside={/*\%}{*/},%
   rangeprefix= /*< ,rangesuffix= >*/,%
   morestring=[d]{"}%
 }

\lstset{breaklines=true,language=scala}

\def\code{\lstinline}  % shorter version so you can write \code|String[Foo]|
                       % -- \def must be in same file as uses for this to
                       % work...
\newcommand{\lstref}[1]{Listing~\ref{#1}}
\newcommand{\Lstref}[1]{Listing~\ref{#1}} % only capitalise at beginning of sentence?
\newcommand{\secref}[1]{Section~\ref{#1}}
\newcommand{\Secref}[1]{Section~\ref{#1}} % only capitalise at beginning of sentence?



% \lstset{basicstyle=\footnotesize\ttfamily, breaklines=true, language=scala, tabsize=2, columns=fixed, mathescape=false,includerangemarker=false}
% thank you, Burak
% (lstset tweaking stolen from
% http://lampsvn.epfl.ch/svn-repos/scala/scala/branches/typestate/docs/tstate-report/datasway.tex)
\lstset{
    xleftmargin=2em,%
    framesep=5pt,%
    frame=none,%
    captionpos=b,%
    fontadjust=true,%
    columns=[c]fixed,%
    keepspaces=false,%
    basewidth={0.56em, 0.52em},%
    tabsize=2,%
    basicstyle=\small\tt,% \small\tt
    commentstyle=\textit,%
    keywordstyle=\bfseries,%
    escapechar=\%,%
}

%% set latex/pdflatex specific stuff
%\ifpdf
%    \usepackage[pdftex,
%                hyperindex,
%                plainpages=false,
%                breaklinks,
%                colorlinks,
%                citecolor=black,
%                filecolor=black,
%                linkcolor=black,
%                pagecolor=black,
%                urlcolor=black]{hyperref}
%    \usepackage[pdftex]{graphicx}
%    \DeclareGraphicsExtensions{.png,.pdf}
%    \pdfcatalog {
%        /PageMode (/UseNone)
%    }
%    \usepackage{thumbpdf}
%    \usepackage[pdftex]{color}
%\else
%    \usepackage[ps2pdf]{hyperref}
%    \usepackage{graphicx}
%    \DeclareGraphicsExtensions{.eps,.jpg}
%    \usepackage{color}
%\fi

%\setlength{\parindent}{0pt}
%\setlength{\parskip}{5pt}

% verbfilter stuff
\newcommand{\prog}[1]{{\sl #1}}
\newenvironment{program}[1][10.5]
  {\fontsize{#1}{13.6}\tt\begin{tabbing}\hspace*{0.5\parindent}\=\+\kill}
  {\end{tabbing}\noindent}
\newcommand{\blockcomment}[1]{{\color{grayPoint3}#1}}
\newcommand{\linecomment}{\color{grayPoint3}}
\newcommand{\grey}{\color{grey}}

%\newenvironment{program}{\ \ \ \ \begin{minipage}{\textwidth}\renewcommand{\baselinestretch}{1.0}\sl\begin{tabbing}}{\end{tabbing}\end{minipage}}
\newcommand{\vem}{\bfseries}
\newcommand{\quotedstring}[1]{{#1}}
\newcommand{\typename}[1]{{#1}}
\newcommand{\literal}[1]{{#1}}

% comments and notes
\newcommand{\comment}[1]{}
%\newcommand{\note}[1]{{\bf $\clubsuit$ #1 $\spadesuit$}}

% figures
\newcommand{\figurebox}[1]
        {\fbox{\begin{minipage}{\textwidth} #1 \medskip\end{minipage}}}
%        {\fbox{\begin{minipage}{\textwidth}\begin{center} #1 \end{center}\medskip\end{minipage}}}
\newcommand{\boxfig}[3]
        {\begin{figure*}\figurebox{#3\caption{\label{fig:#1}#2}}\end{figure*}}
\newcommand{\figref}[1]
        {Figure~\ref{fig:#1}}

% typing rules (not used here)
\newcommand{\ttag}[1]{\mbox{\textsc{\small(#1)}}}
\newcommand{\infer}[3]{\mbox{#1 }\ba{c} #2 \\ \hline #3 \ea}
\newcommand{\irule}[2]{{\renewcommand{\arraystretch}{1.2}\ba{c} #1
                        \\ \hline #2 \ea}}
\newlength{\trulemargin}
\newlength{\trulewidth}
\newlength{\srulewidth}
\setlength{\trulemargin}{0.80cm}
\setlength{\trulewidth}{40.0mm}
\setlength{\srulewidth}{3.0cm}
\newenvironment{trules}{$\vspace{0.5em}\ba{p{\trulemargin}@{~}p{\trulewidth}@{~}p{\trulemargin}}}{\ea$}
\newenvironment{srules}{$\vspace{0.5em}\ba{p{\trulemargin}@{~}p{\srulewidth}}}{\ea$}
\newcommand{\laxiom}[2]{\ttag{#1} & $ #2 \hfill\ }
\newcommand{\raxiom}[2]{\hfill #2 $& \hfill \ttag{#1}}
\newcommand{\caxiom}[2]{\ttag{#1} & $\hfill #2 \hfill $& \ }
\newcommand{\lrule}[3]{\laxiom{#1}{\irule{#2}{#3}}}
\newcommand{\rrule}[3]{\raxiom{#1}{\irule{#2}{#3}}}
\newcommand{\crule}[3]{\caxiom{#1}{\irule{#2}{#3}}}
\newcommand{\lsrule}[3]{\lsaxiom{#1}{\irule{#2}{#3}}}
\newcommand{\rsrule}[3]{\rsaxiom{#1}{\irule{#2}{#3}}}
\newcommand{\nl}{\end{trules}\\[0.5em] \begin{trules}}
\newcommand{\snl}{\end{srules}\\[0.5em] \begin{srules}}

% commas and semicolons
\newcommand{\comma}{,\,}
\newcommand{\commadots}{\comma \ldots \comma}
\newcommand{\semi}{;\mbox{;};}
\newcommand{\semidots}{\semi \ldots \semi}

% math stuff
\newenvironment{myproof}{{\em Proof:}}{$\Box$}
\newenvironment{proofsketch}{{\em Proof Sketch:}}{$\Box$}
\newcommand{\Case}{{\em Case\ }}

% make ; a delimiter in math mode
% \mathcode`\;="8000 % Makes ; active in math mode
% {\catcode`\;=\active \gdef;{\;}}
% \mathchardef\semicolon="003B

% reserved words
\newcommand{\mathem}{\bf}

% brackets
\newcommand{\set}[1]{\{#1\}}
\newcommand{\sbs}[1]{\lquote #1 \rquote}

% arrays
\newcommand{\ba}{\begin{array}}
\newcommand{\ea}{\end{array}}
\newcommand{\bda}{\[\ba}
\newcommand{\eda}{\ea\]}
\newcommand{\ei}{\end{array}}
\newcommand{\bcases}{\left\{\begin{array}{ll}}
\newcommand{\ecases}{\end{array}\right.}

% \cal ids
\renewcommand{\AA}{{\cal A}}
\newcommand{\BB}{{\cal B}}
\newcommand{\CC}{{\cal C}}
\newcommand{\DD}{{\cal D}}
\newcommand{\EE}{{\cal E}}
\newcommand{\FF}{{\cal F}}
\newcommand{\GG}{{\cal G}}
\newcommand{\HH}{{\cal H}}
\newcommand{\II}{{\cal I}}
\newcommand{\JJ}{{\cal J}}
\newcommand{\KK}{{\cal K}}
\newcommand{\LL}{{\cal L}}
\newcommand{\MM}{{\cal M}}
\newcommand{\NN}{{\cal N}}
\newcommand{\OO}{{\cal O}}
\newcommand{\PP}{{\cal P}}
\newcommand{\QQ}{{\cal Q}}
\newcommand{\RR}{{\cal R}}
\newcommand{\TT}{{\cal T}}
\newcommand{\UU}{{\cal U}}
\newcommand{\VV}{{\cal V}}
\newcommand{\WW}{{\cal W}}
\newcommand{\XX}{{\cal X}}
\newcommand{\YY}{{\cal Y}}
\newcommand{\ZZ}{{\cal Z}}

%\newcommand{\inst}{\mbox{\mathem inst}}
\newcommand{\trans}[1]{\la\!\la#1\ra\!\ra}
\newcommand{\remark}[1]{{\bf $\clubsuit$ #1 $\spadesuit$}}
\newcommand{\todo}[1]{\remark{to do: #1}}
%\newcommand{\J}{\justifies}
%\newcommand{\U}{\using}

% names
\newcommand{\Scala}{\mbox{\textsc{Scala}}}
\newcommand{\Java}{\mbox{\textsc{Java}}}

%\renewcommand\textfraction{.05}
%\renewcommand\floatpagefraction{.9}
%\renewcommand\topfraction{.8}

