\title{Safe Low-Overhead Memory Management for Concurrency and Parallelism}
\author{Denys Shabalin}
\date{21 July 2015}
{
\setbeamertemplate{footline}{}
\begin{frame}
    \titlepage
\end{frame}
}


\begin{frame}
    \frametitle{Overview}
    \begin{enumerate}
        \item Region-based memory in Cyclone
        \item Uniqueness and reference immutability for safe parallelism
        \item An efficient on-the-fly cycle collection
        \item Research proposal
    \end{enumerate}
\end{frame}

\begin{frame}
    \begin{center}
        {\LARGE Region-based memory in Cyclone} \\
        \vspace{20pt}
        Dan Grossman, Michael Hicks, Greg Morrisett,\\
        Yanling Wang, Trevor Jim, James Cheney
    \end{center}
\end{frame}

\begin{frame}
    \frametitle{Safe region-based memory management}

    Pioneered by M. Tofte and J.P. Talpin who explored the concept in functional
    programming languages of the ML family.
\end{frame}

\begin{frame}
    \frametitle{Cyclone overview}
    \begin{itemize}
        \item Close to C syntactically to ease porting of existing programs.
        \item Adds tagged unions, polymorphism, \textit{regions} etc.
        \item Guarantees memory safety via static type checking.
    \end{itemize}
\end{frame}

\begin{frame}[fragile]
    \frametitle{Regions in Cyclone}
    \begin{verbatim}
    void main() {
        region r {
            struct Point*r p = rnew(r) { 10.0, 10.0 };
            printf("Point at (%d, %d)", p->x, p->y);
        }
    }
    \end{verbatim}
\end{frame}

\begin{frame}[fragile]
    \frametitle{Regions in Cyclone}
    \begin{verbatim}
    void main() {
        region r {
            struct Point*r p = rnew(r) { 10.0, 10.0 };
            printPoint<r>(p)
        }
    }

    void printPoint<r>(Point*r p) {
        printf("Point at (%d, %d)", p->x, p->y);
    }
    \end{verbatim}
\end{frame}

\begin{frame}[fragile]
    \frametitle{Regions in Cyclone}
    \begin{verbatim}
    void main() {
        region r {
            struct Point*r p = rnew(r) { 10.0, 10.0 };
            printPoint<r>(p)
        }
    }

    void printPoint<r>(Point*r p; {r}) {
        printf("Point at (%d, %d)", p->x, p->y);
    }
    \end{verbatim}
\end{frame}

\begin{frame}[fragile]
    \frametitle{Cyclone type system: abstract syntax excerpt}

    $\tau ::= \tau_1 \xrightarrow{\epsilon} \tau_2\ |
            \ \tau * \rho\ |\ handle(\rho)\ |\ ... $
    \\ \vspace{10pt}

    $e ::= x_\rho\ |\ *e\ |\ rnew(e_1)\ e_2\ |\ ...$
    \\ \vspace{10pt}

    $s ::=  e\ |\ return\ e\ |\ s_1; s_2 |
            \ region \langle \rho \rangle\ x_\rho\ s\ |\ ...$
    \\ \vspace{10pt}

    $\Gamma ::= \bullet\ |\ \Gamma, x_\rho: \tau$
    \\ \vspace{10pt}

    $\Delta ::= \bullet\ |\ \Delta, \alpha: \kappa$
    \\ \vspace{10pt}

    $\gamma ::= \emptyset\ |\ \gamma, \epsilon <: \rho$
    \\ \vspace{10pt}

    $\epsilon ::= \alpha_1\ \cup\ \alpha_2\ \cup \ ...\ \cup\ \alpha_n \ \cup \
                  \{ \rho_1,\ ..., \rho_m \}$
    \\ \vspace{10pt}

    $C ::= \Gamma; \Delta; \gamma; \epsilon$
\end{frame}

\begin{frame}
    \frametitle{Cyclone type system: main judgments}

    $\gamma \ts \epsilon \Rightarrow \rho$
    \\ \vspace{10pt}

    $\gamma \ts \epsilon_1 \Rightarrow \epsilon_2$
    \\ \vspace{10pt}

    $\Delta; \Gamma; \gamma; \epsilon \ts e: \tau$
    \\ \vspace{10pt}

    $\Delta; \Gamma; \gamma; \epsilon; \tau \ts_{stmt}\ s$
\end{frame}

\begin{frame}
    \frametitle{Cyclone type system: typing dereferencing}
    \Large
    \infrule
    {
        \Delta; \Gamma; \gamma; \epsilon \ts e: \tau * \rho \ \ \ \
        \gamma \ts \epsilon \Rightarrow \rho
    }
    {
        \Delta; \Gamma; \gamma; \epsilon \ts *e: \tau
    }
\end{frame}

\begin{frame}
    \frametitle{Cyclone type system: typing applications}
    \Large
    \infrule
    {
        \Delta; \Gamma; \gamma; \epsilon \ts
        e_1: \tau_2 \xrightarrow{\epsilon_1} \tau \ \
        \Delta; \Gamma; \gamma; \epsilon \ts
        e_2: \tau_2 \ \
        \gamma \ts \epsilon \Rightarrow \epsilon_1
    }
    {
        \Delta; \Gamma; \gamma; \epsilon \ts
        e_1 (e_2): \tau
    }
\end{frame}

\begin{frame}
    \frametitle{Cyclone summary}

    \begin{itemize}
        \item Region-based memory management
        \item Statically safe through novel type system
        \item Safety is achieved at expense of user convenience
    \end{itemize}
\end{frame}

\begin{frame}
    \begin{center}
        {\LARGE Uniqueness and reference immutability\\ for safe parallelism} \\
        \vspace{20pt}
        Colin S. Gordon, Matthew J. Parkinson, \\
        Jared Parsons, Aleks Bromfield, Joe Duffy
    \end{center}
\end{frame}

\begin{frame}
    \frametitle{Uniqueness and reference immutability summary}
    \begin{itemize}
        \item Data-race freedom through static checking.
        \item Annotations can also be used for memory management.
        \item Safety comes at expense of notational overhead.
    \end{itemize}
\end{frame}

\begin{frame}
    \begin{center}
        {\LARGE An efficient on-the-fly cycle collection} \\
        \vspace{20pt}
        Harel Paz, David F. Bacon, Elliot K. Kolodner,\\
        Erez Petrank, V. T. Rajan
    \end{center}
\end{frame}

\begin{frame}
    \frametitle{Efficient on-the-fly collection summary}
    \begin{itemize}
        \item Low pause times due to sliding views.
        \item Efficient cycle collection through a single traversal.
        \item Dynamically-sized young generation.
        \item User conevenience at expense of added runtime complexity.
    \end{itemize}
\end{frame}

\begin{frame}
    \frametitle{Summary}
    \Large
    \definecolor{alizarin}{rgb}{0.82, 0.1, 0.26}
    \definecolor{seagreen}{rgb}{0.18, 0.55, 0.34}
    \newcommand{\Best}{{\color{seagreen} Best}}
    \newcommand{\Worst}{{\color{alizarin} Worst}}
    \begin{center}
        \begin{tabular}{c | c | c | c}
                                   & Regions & Ownership & GC     \\
                                   \hline
            Performance            & \Best   & \Best     & \Worst \\
            Predictability         & \Best   & \Best     & \Worst \\
            Notational convenience & \Worst  & \Worst    & \Best

        \end{tabular}
    \end{center}
\end{frame}

\begin{frame}
    \begin{center}
        {\LARGE Research proposal}
    \end{center}
\end{frame}

\begin{frame}
    \frametitle{Research proposal}
    \begin{itemize}
        \item Low-pause on-the-fly GC as a baseline
        \item Optional annotations to hint at desired memory
              management for performance critical sections of code.
        \item Effectively "programmable" garbage collection.
    \end{itemize}
\end{frame}
